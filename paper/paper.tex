
\documentclass{bmcart}

\usepackage{amsthm,amsmath}
\usepackage[utf8]{inputenc}
\usepackage{url}
\usepackage{color}
\usepackage{xspace}

\def\includegraphic{}
\def\includegraphics{}



%%% Put your definitions there:
\startlocaldefs

\newcommand*{\eg}{e.g.\@\xspace}
\newcommand*{\ie}{i.e.\@\xspace}
\newcommand{\ipt}{IPT\xspace}
\newcommand{\ipts}{IPTs\xspace}

\newcommand{\patk}{p\mathrm{@}k}

\newcommand{\ml}[1]{\textcolor{blue}{ML: #1}}
\newcommand{\vl}[1]{\textcolor{red}{VL: #1}}
\newcommand{\ja}[1]{\textcolor{purple}{JA: #1}}

\endlocaldefs


%%% Begin ...
\begin{document}

%%% Start of article front matter
\begin{frontmatter}

\begin{fmbox}
\dochead{Research}

%%%%%%%%%%%%%%%%%%%%%%%%%%%%%%%%%%%%%%%%%%%%%%
%%                                          %%
%% Enter the title of your article here     %%
%%                                          %%
%%%%%%%%%%%%%%%%%%%%%%%%%%%%%%%%%%%%%%%%%%%%%%

\title{Learning Auditory Similarities Between Instrumental Playing Techniques}

%%%%%%%%%%%%%%%%%%%%%%%%%%%%%%%%%%%%%%%%%%%%%%
%%                                          %%
%% Enter the authors here                   %%
%%                                          %%
%% Specify information, if available,       %%
%% in the form:                             %%
%%   <key>={<id1>,<id2>}                    %%
%%   <key>=                                 %%
%% Comment or delete the keys which are     %%
%% not used. Repeat \author command as much %%
%% as required.                             %%
%%                                          %%
%%%%%%%%%%%%%%%%%%%%%%%%%%%%%%%%%%%%%%%%%%%%%%

\author[
   addressref={aff1},                   % id's of addresses, e.g. {aff1,aff2}
   corref={aff1},                       % id of corresponding address, if any
   noteref={n1},                        % id's of article notes, if any
   email={email@example.com}            % email address
]{\inits{CE}\fnm{Christian} \snm{El-Hajj}}
\author[
   addressref={aff2},
   email={email@example.com}
]{\inits{VL}\fnm{Vincent} \snm{Lostanlen}}
\author[
   addressref={aff2},
   email={email@example.com}
]{\inits{MR}\fnm{Mathias} \snm{Rossignol}}
\author[
   addressref={aff2},
   email={email@example.com}
]{\inits{GL}\fnm{Gr\'egoire} \snm{Lafay}}
\author[
   addressref={aff2},
   email={email@example.com}
]{\inits{ML}\fnm{Mathieu} \snm{Lagrange}}

%%%%%%%%%%%%%%%%%%%%%%%%%%%%%%%%%%%%%%%%%%%%%%
%%                                          %%
%% Enter the authors' addresses here        %%
%%                                          %%
%% Repeat \address commands as much as      %%
%% required.                                %%
%%                                          %%
%%%%%%%%%%%%%%%%%%%%%%%%%%%%%%%%%%%%%%%%%%%%%%

\address[id=aff1]{%                           % unique id
  \orgname{Department of Zoology, Cambridge}, % university, etc
  \street{Waterloo Road},                     %
  %\postcode{}                                % post or zip code
  \city{London},                              % city
  \cny{UK}                                    % country
}
\address[id=aff2]{%
  \orgname{Marine Ecology Department, Institute of Marine Sciences Kiel},
  \street{D\"{u}sternbrooker Weg 20},
  \postcode{24105}
  \city{Kiel},
  \cny{Germany}
}

%%%%%%%%%%%%%%%%%%%%%%%%%%%%%%%%%%%%%%%%%%%%%%
%%                                          %%
%% Enter short notes here                   %%
%%                                          %%
%% Short notes will be after addresses      %%
%% on first page.                           %%
%%                                          %%
%%%%%%%%%%%%%%%%%%%%%%%%%%%%%%%%%%%%%%%%%%%%%%

\begin{artnotes}
\note[id=n1]{Equal contributor}
\end{artnotes}

\end{fmbox}

%%%%%%%%%%%%%%%%%%%%%%%%%%%%%%%%%%%%%%%%%%%%%%
%%                                          %%
%% The Abstract begins here                 %%
%%                                          %%
%% Please refer to the Instructions for     %%
%% authors on http://www.biomedcentral.com  %%
%% and include the section headings         %%
%% accordingly for your article type.       %%
%%                                          %%
%%%%%%%%%%%%%%%%%%%%%%%%%%%%%%%%%%%%%%%%%%%%%%

\begin{abstractbox}

\begin{abstract}
Abstract here.
\end{abstract}

%%%%%%%%%%%%%%%%%%%%%%%%%%%%%%%%%%%%%%%%%%%%%%
%%                                          %%
%% The keywords begin here                  %%
%%                                          %%
%% Put each keyword in separate \kwd{}.     %%
%%                                          %%
%%%%%%%%%%%%%%%%%%%%%%%%%%%%%%%%%%%%%%%%%%%%%%

\begin{keyword}
\kwd{sample}
\kwd{article}
\kwd{author}
\end{keyword}

\end{abstractbox}

\end{frontmatter}

%%%%%%%%%%%%%%%%%%%%%%%%%%%%%%%%%%%%%%%%%%%%%%
%%                                          %%
%% The Main Body begins here                %%
%%                                          %%
%% Please refer to the instructions for     %%
%% authors on:                              %%
%% http://www.biomedcentral.com/info/authors%%
%% and include the section headings         %%
%% accordingly for your article type.       %%
%%                                          %%
%% See the Results and Discussion section   %%
%% for details on how to create sub-sections%%
%%                                          %%
%% use \cite{...} to cite references        %%
%%  \cite{koon} and                         %%
%%  \cite{oreg,khar,zvai,xjon,schn,pond}    %%
%%  \nocite{smith,marg,hunn,advi,koha,mouse}%%
%%                                          %%
%%%%%%%%%%%%%%%%%%%%%%%%%%%%%%%%%%%%%%%%%%%%%%

\section*{Introduction}
\label{sec:intro}

% Section outline:
% - How to accurately model timbre space? Useful for studying auditory perception, browsing databases, etc.
% - Studies either consider stimuli that are ``too simple'' (e.g., tones, instruments with fixed playing techniques) or ``too complex'' (e.g., speech, environmental sounds).
% - Computational models do not capture enough information (MFCCs and related representations) or are hard to compute or analyze (STRFs, deep networks).
% - Present a study of timbral similarity perception on diverse set of instrumental sounds with extended playing techniques (PTs). Provides more control in modeling different aspects of timbre.
% - In addition, propose a computational model for timbral similarity: scattering (separable or joint) plus a linear layer. Fixed representation to extract time-frequency structure with adaptivity in linear component.

One of the most important properties for identifying the source of a sound is its \emph{timbre}.
While it is typically defined as the quality of a sound that is independent of its pitch, duration, or loudness, this definition fails to provide an explicit characterization of the attribute.
Due to its central role in audio perception, timbre has been studied from several angles, including perceptual and psychophysical studies.
Since timbre extraction forms the basis of many audio-based information retrieval tasks, such as classification and determining acoustical similarity, the subject has also been studied from a machine listening perspective.
\ja{Need references for the above statements.}

A natural candidate for characterizing timbre is the spectral content of a signal.
This information, however, is not rich enough to fully characterize the timbre of a sound, as evidenced by the recent move from spectral descriptors such as mel-spectrograms and mel-frequency cepstral coefficients (MFCCs) and descriptors of spectral energy distributions.
Indeed, the success of deep neural networks (DNNs) applied to time--frequency representations (such as spectrograms or scalograms) in machine listening have demonstrated the necessity for capturing both spectral and temporal content.
In neurophysiological, the auditory perception models proposed by S. Shamma and collaborators, known as \emph{spectro--temporal receptive fields} (STRFs) all characterize temporal structure in addition to purely spectral information.
It seems, therefore, that timbre is inherently spectro--temporal.
\ja{Again, refs.}

While the above models offer some insight into the structure of timbre, their complexity makes interpretation and analysis difficult.
The expressive power of DNNs are due to tuning a large number of parameters, so it is often hard to relate a specific component to a property of the resulting perception model.
STRFs, on the other hand, have a simpler structure, but are ultimately constructed to model electrophysiological signal in the auditory cortex.
As a result, they consist several adjustments that make their computation costly and their mathematical analysis difficult.
\ja{Refs.}

Another difficulty in studying the perception of timbre is the choice of stimuli.
Previous work on timbre characterization of STRFs \cite{patil2012music} has focused on instrumental sounds of isolated notes played with standard playing techniques.
This inherently restricts the range of available timbre studied, limiting the conclusions of the work.
On the other end of the spectrum are computational models for information retrieval tasks.
Since these are primarily concerned with performance on real-world sounds, the possible range of timbre is very broad.
The analysis of the resulting model is therefore all the more difficult, since we have little control over the complexity of the stimuli.
\ja{Refs.}

Our contribution is twofold.
First, we propose a simple model for timbre perception involving a fixed time--frequency descriptor, the scattering, followed by a learned linear layer optimized for the timbral space under consideration.
Second, we validate this model on a set of instrumental sounds with extended playing techniques (PTs), allowing for a richer range of timbre compared to ordinary instrumental sounds while retaining the structure of the space necessary for interpretability.

As part of validating the model, we have collected timbral similarity judgments in a crowdsourced listening experiment.
These are used as a ``ground truth'' dataset, which allow us to evaluate the model's ability to characterize the perceived similarity of the stimuli.
Analysis of these experimental results also confirms the necessity of including a range of PTs, as these are shown to influence similarity judgments to the same extent as the identity of the instrument itself.

We conclude with a discussion of potential applications to browsing interfaces for musical databases, where composers can easily find sounds on a query-by-example basis.
The adaptivity of the proposed model has the additional advantag of allowing it to be adjusted to a particular listener's preferences by retraining the linear layer.

\section*{Timbre similarity judgments}
\label{sec:survey}

To validate our timbre perception model, we first acquired similarity judgments from a survey of human subjects.
The survey was carried out for $78$ sounds, each representing an instrument--playing technique (IPT) pair.
These IPTs were selected from a larger set of $235$ IPTs by two experts at the \emph{Conservatoire national sup\'erieur de musique et de danse de Paris} (CNSMDP).
The selection criteria were based on the likelihood of an IPT being similar to some other IPT, ensuring that IPTs that were too singular were excluded from the survey.

In the survey, a subject was provided with these $78$ IPTs at the same pitch and asked to cluster them in a free sorting task.
This was achieved by arranging correspoding dots in a two-dimensional plane on a computer screen and assigning a color to each dot.
While the location of the dots was recorded, this information was not used in our analysis and only served to help the subject organize the sounds spatially.
The colors assigned to each dot---and, consequently, to each sound---was recorded for each subject.

The above similarity judgments were crowdsourced by sending out requests to perform the experiment on internal mailing lists of the CNSMDP and international research mailing lists with focus on audio and music processing.
We made the experiment available for two months and obtained results for a total of $31$ subjects.

Given the clustering assignments of each subject, we then constructed a consensus clustering using a hypergraph partitioning algorithm \cite{kernighan1970efficient,han1997scalable,strehl2002cluster}.
Interestingly, although several sounds pairs were obtained from the same instrument (but different PTs) and others were obtained using the same PT (but different instruments), the consensus clustering did not favor one of these taxonomies over the other.

\section*{Proposed approach}
\label{sec:method}

\subsection*{Scattering transform}
\label{sec:scattering}

While cortical representations have been successful in distinguishing properties of audio recordings (including similarity analysis and classification tasks), they remain models for the electrophysiological signals recorded in the audio cortex.
As such, their calculation is not always computationally efficient and their mathematical properties may be difficult to analyze directly.

A closely related representation is that of the time scattering transform, introduced by And\'{e}n and Mallat \cite{anden2011multiscale,anden2014deep}.
Given an audio recording, it decomposes the signal using an analytic wavelet transform and computes the modulus, obtaining the \emph{scalogram} of the signal.
The scalogram is similar to the spectrogram, but has a time--frequency resolution that varies with frequency: at low frequency, the filters are narrowly tuned in frequency (and hence wide in time), while at high frequency, the filters are wide in frequency (and hence narrowly concentrated in time).
This representation contains a large amount of information, but not in a form suitable for building statistical models.
In particular, the scalogram is not \emph{stable} in the sense that small deformations to the original signal may induce large changes to the scalogram.
It also lacks \emph{invariance} to time-shifting, which does not affect our perception of a sound, and should therefore not affect its timbre model.
One way to resolve this is to average the scalogram in time to obtain a spectral profile of the signal, yielding the first-order scattering coefficients.
These possess both invariance to time-shifting and stability to time-warping deformations \cite{anden2014deep}.

Since these first-order coefficients only include spectral information, they cannot accurately characterize more sophisticated structure in the signal.
One way to augment them is to calculate another wavelet decomposition in time, this time on each frequency channel of the scalogram.
Again, we compute the modulus of the wavelet coefficients and average in time.
The resulting coefficients are known as the second-order time scattering coefficients.
These also possess the desired invariance and stability properties, but capture more information on the temporal structure of the scalogram \cite{anden2014deep}.

While richer than a simple spectral decomposition, such as MFCCs, the second-order time scattering coefficients described above primarily capture structure in time.
Indeed, they do not explicitly encode frequency structure in an invariant and stable manner.
If a signal is transposed or warped in frequency by a small amount, we do not expect its timbre to change significantly.
We would therefore like our representation to have the same invariance and stability properties along the frequency axis.
This is achieved by taking first- and second-order scattering coefficients and computing a second scattering transform, this time along the log-frequency axis.
The resulting coefficients are known as separable time and frequency scattering coefficients \cite{anden2014deep}.

This representation does possess the desired invariance and stability properties, but its separable nature limits its power to adequately represent time--frequency structures that do not separate cleanly in time and frequency, such as chirps.
To remedy this, another approach was suggested by And\'{e}n et al. \cite{anden2015joint,anden2019joint} wherein the temporal wavelet decomposition of the scalogram is replaced with a two-dimensional, spectro-temporal decomposition.
Similarly to the cortical representation, the result is a four-dimensional array indexed by time, acoustic frequency, modulation frequency (or \emph{rate}), and spectral modulation frequency, known as \emph{quefrency} (or \emph{scale}).
As before, we compute the modulus and average in time, obtaining a set of coefficients known as second-order joint time--frequency scattering coefficients.

The definition of the scattering transform in terms of wavelet decompositions and modulus nonlinearities presents several advantages.
On the computational side, the critical bandwidth guarantees of wavelet transforms allows us to judiciously subsample the output, resulting in a lower-dimensional representation compared to the cortical representations.
The scattering transform also satisfies the aforementioned stability conditions, providing a guarantee that slightly deforming a signal only results in a negligible change in its scattering transform.
Furthermore, we may calculate the scattering tranforms of several model signals, including amplitude-modulated harmonic sounds and noise excitations \cite{anden2012scattering,anden2014deep}, frequency-modulated sinusoids (e.g., chirps) \cite{anden2012scattering,anden2019joint}, and beating phenomena \cite{anden2014deep}.
They have also enjoyed success in several audio classification and similarity retrieval tasks \cite{anden2011multiscale,bauge2013representing,anden2014deep,anden2019joint,lostanlen2018relevance,lostanlen2018extended}.

To evaluate the ability of the scattering transform to characterize the timbre of an audio signal, we consider the task of classifying musical notes from a wide variety of instruments played according to a set of extended instrumental playing techniques (IPTs).
Increasing the variability in IPTs allows us to explore a wider range of timbre, providing a more robust validation of the proposed model for modeling timbre.
The metric used is the \emph{precision at rank $k$} ($\patk$), which is calculated by computing the $k$ nearest neighbors in the feature space and recording the proportion of the neighbors that are in the same class.
In the following, we set $k = 5$.
The classes were obtained from the consensus clustering of human similarity judgments described in the previous section.
The $\patk$ therefore provides a measure of consistency between the timbre similarity measures obtained by our model and those of the survey subjects.

Concatenating the first- and second-order separable time and frequency scattering coefficients, we obtain a feature vector of dimension $???$ for a sound recording of duration $???$.
Computing nearest neighbors with the standard Euclidean measure gives a $\patk$ of $0.870 \pm 0.058$, where the first value indicates the mean and the second the standard deviation.
As a result, this simple transformation of the original signal results in a representation that lends itself naturally to discriminating timbre.
Note that this result is obtained by simply looking at the raw feature vectors---no learning is performed to adapt the transform towards discriminating timbre.
Replacing the separable second-order coefficients with the joint second-order coefficients boosts the $\patk$ to $???$.
This is evidence that accurately characterizing the joint time--frequency structure of the sound is important capturing its timbre.

We performed the same evaluation with a standard MFCC representation, obtaining a $\patk$ of $0.851 \pm 0.062$.
As a result, we conclude that a Euclidean distance over scattering transforms (separable or joint) yield little additional perceptual information compared to standard spectral descriptors.

\subsection*{Similarity weighting}
\label{sec:weighting}

To improve our measure, we need to replace the standard Euclidean distance on the raw scattering coefficients by something better adapted to discriminating timbre.
Equivalently, we can apply further transformations to the scattering coefficients and compute the Euclidean distance on the result.
Very flexible and powerful methods can be brought to bear on this problem, such as deep neural networks, but with their flexibility comes an increase in complexity.
This increased complexity of the model reduces its usefulness with respect to interpretability.

For this reason, we restrict ourselves to a simple linear transform of the scattering coefficients.
Specifically, we construct a positive semidefinite matrix $L$ that we apply to the scattering coefficients $x$ so that the Euclidean distance between $Lx$ and $Ly$ best approximates the consensus clustering obtained above.
We expect the resulting representation $Lx$ to provide a more accurate model for timbre perception.

We now use the consensus clustering to reweight the similarity between feature vectors (either MFCCs or scattering transforms).
For this, we use the \emph{large margin nearest neighbor} (LMNN) metric learning algorithm \cite{weinberger2006distance, weinberger2009distance}.
This approach constructs a positive semi-definite weighting matrix $L$ such that the distance $\|Lx - Ly\|$ best classifies a set of points into a fixed clustering assignment.
Running LMNN on MFCCs with the consensus clustering obtained above increases $\patk$ slightly to $0.862 \pm 0.059$.
Consequently, there seems to be little additional timbre information in the MFCCs compared to that given by the standard Euclidean distance.
We contrast this with LMNN applied to the separable scattering coefficients, which yields a $\patk$ of $0.948 \pm 0.033$, which is a significant increase.
For the joint scattering coefficients, we obtain a $\patk$ of $???$.
We therefore conclude that a simple reweighting of the scattering transform yields a more appropriate timbral similarity measure.

\section*{Results}
\label{sec:results}

\ja{The results are currently sprinkled into the above methods section.
These need to be extracted and summarized here.}

\section*{Discussion}
\label{sec:discussion}

\begin{itemize}
\item Demonstrates complex timbral structure perceived by subjects in study: neither instrument (i.e., spectral envelope) or playing technique (i.e., temporal modulation) dominates similarity judgment. Need joint to characterize time-frequency structure (see Patil et al).
\item Proposed computational model (scattering + linear) reproduces similarity judgments accurately. Easy to ``retrain'' for other data. Fixed wavelet structure allows for analysis of linear layer and guarantees invariance and stability properties.
\item Applications: query-by-example in musical sample databases.
\end{itemize}


%%%%%%%%%%%%%%%%%%%%%%%%%%%%%%%%%%%%%%%%%%%%%%
%%                                          %%
%% Backmatter begins here                   %%
%%                                          %%
%%%%%%%%%%%%%%%%%%%%%%%%%%%%%%%%%%%%%%%%%%%%%%

\begin{backmatter}

\section*{Competing interests}
  The authors declare that they have no competing interests.

\section*{Author's contributions}
    Text for this section \ldots

\section*{Acknowledgements}
  Text for this section \ldots
%%%%%%%%%%%%%%%%%%%%%%%%%%%%%%%%%%%%%%%%%%%%%%%%%%%%%%%%%%%%%
%%                  The Bibliography                       %%
%%                                                         %%
%%  Bmc_mathpys.bst  will be used to                       %%
%%  create a .BBL file for submission.                     %%
%%  After submission of the .TEX file,                     %%
%%  you will be prompted to submit your .BBL file.         %%
%%                                                         %%
%%                                                         %%
%%  Note that the displayed Bibliography will not          %%
%%  necessarily be rendered by Latex exactly as specified  %%
%%  in the online Instructions for Authors.                %%
%%                                                         %%
%%%%%%%%%%%%%%%%%%%%%%%%%%%%%%%%%%%%%%%%%%%%%%%%%%%%%%%%%%%%%

\bibliographystyle{bmc-mathphys}
\bibliography{bib}

%%%%%%%%%%%%%%%%%%%%%%%%%%%%%%%%%%%
%%                               %%
%% Figures                       %%
%%                               %%
%% NB: this is for captions and  %%
%% Titles. All graphics must be  %%
%% submitted separately and NOT  %%
%% included in the Tex document  %%
%%                               %%
%%%%%%%%%%%%%%%%%%%%%%%%%%%%%%%%%%%

\section*{Figures}

\begin{figure}[h!]
\caption{\csentence{STRF + scattering (upsampling STRF)}
A short description of the figure content should go here.}
\end{figure}

\begin{figure}[h!]
\caption{\csentence{Model overview}
One line music players (3 ipts bass pizz, bass archo, violin archo) sound, spec,
scat, isomap.
Second line : perceptual clustering.}
\end{figure}

\begin{figure}[h!]
\caption{\csentence{Sample scattering}
Same 3 sounds spec + scatt.}
\end{figure}

\begin{figure}[h!]
\caption{\csentence{Tag clouds}
A short description of the figure content should go here.}
\end{figure}

\begin{figure}[h!]
\caption{\csentence{Perfs MFCC, separable, joint (last)}
A short description of the figure content should go here.}
\end{figure}

%%%%%%%%%%%%%%%%%%%%%%%%%%%%%%%%%%%
%%                               %%
%% Tables                        %%
%%                               %%
%%%%%%%%%%%%%%%%%%%%%%%%%%%%%%%%%%%

\section*{Tables}
\begin{table}[h!]
\caption{Sample table title. This is where the description of the table should go.}
      \begin{tabular}{cccc}
        \hline
           & B1  &B2   & B3\\ \hline
        A1 & 0.1 & 0.2 & 0.3\\
        A2 & ... & ..  & .\\
        A3 & ..  & .   & .\\ \hline
      \end{tabular}
\end{table}

%%%%%%%%%%%%%%%%%%%%%%%%%%%%%%%%%%%
%%                               %%
%% Additional Files              %%
%%                               %%
%%%%%%%%%%%%%%%%%%%%%%%%%%%%%%%%%%%

\section*{Additional Files}
  \subsection*{Additional file 1 --- Sample additional file title}
    Additional file descriptions text (including details of how to
    view the file, if it is in a non-standard format or the file extension).  This might
    refer to a multi-page table or a figure.

  \subsection*{Additional file 2 --- Sample additional file title}
    Additional file descriptions text.

\end{backmatter}

\end{document}
