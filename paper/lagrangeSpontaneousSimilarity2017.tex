% -----------------------------------------------
% Template for ISMIR Papers
% 2017 version, based on previous ISMIR templates

% Requirements :
% * 6+n page length maximum
% * 4MB maximum file size
% * Copyright note must appear in the bottom left corner of first page
% * Clearer statement about citing own work in anonymized submission
% (see conference website for additional details)
% -----------------------------------------------

\documentclass{article}
\usepackage{amsmath,cite,url}
\usepackage{graphicx}
\usepackage{color}


% Title.
% ------
\title{Spontaneous Similarity Judgments of Musical Tones played with various Playing Techniques}

% Note: Please do NOT use \thanks or a \footnote in any of the author markup

% Single address
% To use with only one author or several with the same address
% ---------------
%\oneauthor
% {Names should be omitted for double-blind reviewing}
% {Affiliations should be omitted for double-blind reviewing}

% Two addresses
% --------------
%\twoauthors
%  {First author} {School \\ Department}
%  {Second author} {Company \\ Address}

%% To make customize author list in Creative Common license, uncomment and customize the next line
%  \def\authorname{First Author, Second Author}


% Three addresses
% --------------
\author{Mathieu Lagrange, Christian EL-Hajj, Mathias Rossignol, Grégoire Lafay}

%% To make customize author list in Creative Common license, uncomment and customize the next line
%  \def\authorname{First Author, Second Author, Third Author}

% Four or more addresses
% OR alternative format for large number of co-authors
% ------------
%\multauthor
%{First author$^1$ \hspace{1cm} Second author$^1$ \hspace{1cm} Third author$^2$} { \bfseries{Fourth author$^3$ \hspace{1cm} Fifth author$^2$ \hspace{1cm} Sixth author$^1$}\\
%  $^1$ Department of Computer Science, University , Country\\
%$^2$ International Laboratories, City, Country\\
%$^3$  Company, Address\\
%{\tt\small CorrespondenceAuthor@ismir.edu, PossibleOtherAuthor@ismir.edu}
%}
%\def\authorname{First author, Second author, Third author, Fourth author, Fifth author, Sixth author}


\sloppy % please retain sloppy command for improved formatting

\begin{document}
%
\maketitle
%
\begin{abstract}

Musical timbre is a multi-faceted notion that have been extensively studied, mostly focusing on its spectral aspect. By studying the spontaneous judgments of similarity between several instruments played with a diverse set of playing techniques, we aim in this paper at better understanding the influence of the change of the playing technique on the perception of timbre.

A first experiment is conducted by music experts to identify which couple of instrument and playing technique in a given dataset is worth comparing to other couples of instrument / playing technique. In a second experiment, spontaneous similarity judgments among the retained couples are collected using a canonic free sorting task experiment design.


To further study the outcomes of this experiment, we assume a two-step computational model of human perception, where the acoustic signal is fed to a statically designed processing unit that accounts for frequency and temporal modulations. The resulting features are then projected using a supervised technique which considers as input the perceptual similarity judgments gathered in the second experiment. Numerical experiments show that the induced perceptual space is able to approximate perceptual data with satisfying accuracy.

\end{abstract}
%
\section{Introduction}\label{sec:introduction}

Musical timbre is an interesting notion to study as it is perceptively defined. Thus, by studying musical timbre, one study many facets of the human auditory system which responds to a diverse -- but somewhat more controled than other stimuli such as speech or environemntal sounds -- set of physical stimuli.

A popular view of the kind of stimuli we consider is that they can be described by a spread of energy that evolve through time and across frequency. By following this trend, most researchers translate the negation of the ASA definition of timbre into a more relative definition : musical timbre is about the distribution of intensity across the time / frequency plane.

A great deal of litterature focuses on the instaneous spectrum, \textit{i.e.} the timbre can be defined without considering evolution of the energy distribution trough time. Other studies demonstrated the importance of the onset in the recognition of some musical instruments. Those studies on a rather short term model where the waveform is observed in a range that do not exceed 100 ms.

Interestingly, modern physiological models of the primary mamalian auditory system also considers the evolution of the energy distribution at longer time scales, \textit{i.e.} how the energy modulates through time for given frequencies.  Dau's model acount for frequency and rate of modulation. Shamma's model account for frequency, rate and bandwidth to fully describes the physiological evidence gathered by studying the auditory system of the ferret. Though, most of the numerical experiments done using this model did not fully demonstrated the usefullness of this third dimension.

Thus, by taking a signal processing view of the matter, one could consider that a first stage of the auditory system proceeds to a rather fixed set of convolutions across time and frequency, projecting the signal into a large set of descriptive features organized across frequency and rate of modulations.

What the higher level of the auditory cortex does remains largely unkown, but we can assume that the level of plasticity is very high, as any living entity has to adapt their percepts to a wide diversity of tasks.  Considering the lack of information about this stage, we will assume in this paper that this second step is rather opportunistic, in the sense that, from this large set of features, it aims at multiplexing in some way the ones that are the most relevant for the task at hand.

Considering not only the type of musical instrument but also the playing technique is interesting in that matter, as it invite us to consider the relation between the rate of modulation and timbre percepts. A notion that have not been extensively studied in the litterature. It also permits to consider the notion of timbral similarity at finer grain of detail.

The contributions of the paper are as follows: 1) provide perceptual data that account for the perception of musical tones played with various playing techniques, 2), demonstrate that, for the gathered data, the type of playing technique plays an important role for the organization of the data, and 3) propose a perceptually motivated computational model of timbre perception that account well for the human judgments studied in this paper.



\section{Experiment 1}\label{sec:}


The dataset considered in this study has audio recording of 16 different  musical instruments () played with different playing techniques, which leads to 143 different couple of instrument / playing technique. For each couple, the pitch and intonation is varied leading to 25444 samples. The way the dataset is segmented is explained in more details in Appendix \ref{sec:dataset}.

Studies about perceptual similarity are plagued with a dimensional problem. When considering $n$ items, a complete exploration of the similarity space requires the filling of a $n^2$ matrix. Assuming a symmetric similarity (the similarity of A to B is equl to the similarity of B to A) reduces the number but not by much.

In an attempt to reduce the number of items, we decided to perform a selection experiment where the subject is asked to give his opinion on which \ipt is relevant to study. An \ipt is said to be relevant to study if it is likely to be associated to another \ipt. Interest is rated on a 7 ticks scale.

Examples are given :
\begin{itemize}
  \item One star: this \ipt is singular, it is not useful to compare it with another \ipt.
  \item Four stars: there is a proximity on one aspect of the sound between this \ipt and of an \ipt of another instrument, but this one is neither decisive nor obvious.
  \item Seven stars: there is a large similarity between this \ipt and an \ipn of another instrument, it is interesting to keep it.
\end{itemize}


Select which couple (instrument, playing technique) are interesting to consider.
Question: Is it interesting to compare this couple to another couple of a different instrument ?
The question is asked for each couple ()
The subject can listen to all the recordings of each couple


Two musical composition professors of the CNSMDP, a renowned French school of music did the test. All the couples with a grade higher than three are selected, bringing 78 couples to study.

The aim of experiment 1

Problématique

Dans le cadre du projet TICEL, nous allons construire un algorithme permettant de mesurer les similarités entre différents modes de jeu. Par mode de jeu, nous entendons ici un couple "instrument + mode de jeu".

Afin de mesurer les performances de l'algorithme, nous avons besoin d'établir une vérité de terrain, en l’occurrence les similarités entre modes de jeu données par des humains.

Or il n'est actuellement pas envisageable de monter une expérience permettant d'établir cette vérité terrain. Le nombre de modes de jeu à comparer est trop important.

C'est pourquoi nous proposons une expérience intermédiaire ayant pour but de réduire le nombre de modes de jeu.

Objectif

Pour chaque mode de jeu, vous répondrez à la question suivante : Est il intéressant de comparer ce mode de jeu à un/pls autre(s) mode(s) d'un autre instrument

Procédure

Nous vous demandons de noter chaque mode de jeu avec un nombre d’étoiles. Plus le nombre d'étoiles est important, plus le mode de jeu est susceptible d'être associé de manière "analogique" à un autre mode de jeu d'un autre instrument. Pour exemple :

Une étoile : ce mode de jeu est singulier, il n’est pas utile de le comparer avec un autre mode de jeu d'un autre instrument.
Quatre étoiles : il y a bien une proximité sur un aspect du son entre ce mode de jeu et un autre mode d'un autre instrument, mais cette dernière n'est ni décisive ni évidente.
Sept étoiles : il a une grande proximité entre ce mode de jeu et un autre mode d'un autre instrument, il est donc intéressant de le conserver.
Déroulement

Vous pouvez réaliser l'expérience en plusieurs fois, mais veillez à noter l'ensemble des modes de jeu pour tous les instruments.

Usage

Votre notation doit prendre en compte l'ensemble des modes de jeu présents dans la base, bien que par souci de clarté, les modes de jeu soient présentés par instruments
Les modes de jeu déjà notés apparaîtront en vert
Pour chaque mode de jeu, vous pouvez écouter plusieurs enregistrements à différentes hauteurs et nuances



\section{Experiment 2}\label{sec:}

\section{Analysis}\label{sec:}

\section{Model}\label{sec:}

\section{Results}\label{sec:}

\section{Discussion}\label{sec:}




% For bibtex users:
\bibliography{bib}

\Appendix

\section{Dataset}


% For non bibtex users:
%\begin{thebibliography}{citations}
%
%\bibitem {Author:00}
%E. Author.
%``The Title of the Conference Paper,''
%{\it Proceedings of the International Symposium
%on Music Information Retrieval}, pp.~000--111, 2000.
%
%\bibitem{Someone:10}
%A. Someone, B. Someone, and C. Someone.
%``The Title of the Journal Paper,''
%{\it Journal of New Music Research},
%Vol.~A, No.~B, pp.~111--222, 2010.
%
%\bibitem{Someone:04} X. Someone and Y. Someone. {\it Title of the Book},
%    Editorial Acme, Porto, 2012.
%
%\end{thebibliography}

\end{document}
